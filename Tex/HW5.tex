\documentclass[11pt]{article}
\usepackage[utf8]{inputenc} % Para caracteres en espa�ol
\usepackage{amsmath,amsthm,amsfonts,amssymb,amscd}
\usepackage{multirow,booktabs}
\usepackage[table]{xcolor}
\usepackage{fullpage}
\usepackage{lastpage}
\usepackage{enumitem}
\usepackage{multicol}
\usepackage{fancyhdr}
\usepackage{mathrsfs}
\usepackage{pdfpages}
\usepackage{wrapfig}
\usepackage{setspace}
\usepackage{esvect}
\usepackage{calc}
\usepackage{multicol}
\usepackage{cancel}
\usepackage{graphicx}
\graphicspath{ {pictures/} }
\usepackage[retainorgcmds]{IEEEtrantools}
\usepackage[margin=3cm]{geometry}
\usepackage{amsmath}
\newlength{\tabcont}
\setlength{\parindent}{0.0in}
\setlength{\parskip}{0.05in}
\usepackage{empheq}
\usepackage{framed}
\usepackage[most]{tcolorbox}
\usepackage{xcolor}
\colorlet{shadecolor}{orange!15}
\parindent 0in
\parskip 12pt
\geometry{margin=1in, headsep=0.25in}
\theoremstyle{definition}
\newtheorem{defn}{Definition}
\newtheorem{reg}{Rule}
\newtheorem{exer}{Exercise}
% Two more packages that make it easy to show MATLAB code
\usepackage[T1]{fontenc}
\usepackage[framed,numbered]{matlab-prettifier}
\lstset{
	style = Matlab-editor,
	basicstyle=\mlttfamily\small,
}
\newtheorem{note}{Note}
\begin{document}  
\setcounter{section}{0}
\thispagestyle{empty}

\begin{lstlisting}
%%%%%%%%%%%%%%%%%%%%%%%%%%%%%%%%%%%%%%%%%%%%%%%%%%%%%%%%%%%%%%%%%%%%%%%%%%%
%% Problem 1
%%%%%%%%%%%%%%%%%%%%%%%%%%%%%%%%%%%%%%%%%%%%%%%%%%%%%%%%%%%%%%%%%%%%%%%%%%%
function P1
%% In EMOS Units
	r1 = 1; % Mars Semi-Major Axis [AU]
	r2 = 1.5237; % Mars Semi-Major Axis [AU]
	mu = 1; % Standard Gravitation of Sun [AU^3/TU^2]

	disp('Normalized Units')
	[a_H,t_H,dv_1, dv_2, dv_tot] = HohmannTransfer(r1,r2,mu);


	mfmo = exp(-dv_tot/0.1);
	mpmo = 1-mfmo;

	eps = 1/7;
	mo=100;
	mp = 84.71;

	ms = (eps*mp)/(1-eps);
	ml = 100-(mp+ms);

%% In Standard Units
	r1 = 149600000; % Earth Semi-Major Axis [km]
	r2 = 227940000; % Mars Semi-Major Axis [km]
	mu = 1.327e11; % Standard Gravitation of Sun [km^3/s^2]

	disp('Standard Units')
	[a_H,t_H,dv_1, dv_2, dv_tot] = HohmannTransfer(r1,r2,mu);

end

%%%%%%%%%%%%%%%%%%%%%%%
% Hohmann Transfer
%%%%%%%%%%%%%%%%%%%%%%%

function [a_H,t_H,dv_1, dv_2, dv_tot] = HohmannTransfer(r1,r2,mu)
	a_H = (r1+r2)/2;
	t_H = pi*sqrt((a_H^3)/mu);

	dv_1 = sqrt(mu*((2/r1)-(1/a_H))) - sqrt(mu/r1);
	dv_2 = sqrt(mu/r2) - sqrt(mu*((2/r2)-(1/a_H)));
	dv_tot = abs(dv_1) + abs(dv_2);
end

\end{lstlisting}
\newpage
%%%%%%%%%%%%%%%%%%%%%%%%%%%%%%%%%%%%%%%%%%%
%%%%%%%%%%%%%%%%%%%%%%%%%%%%%%%%%%%%%%%%%%%
\begin{lstlisting}
%%%%%%%%%%%%%%%%%%%%%%%%%%%%%%%%%%%%%%%%%%%%%%%%%%%%%%%%%%%%%%%%%%%%%%%%%%%
%% Problem 2
%%%%%%%%%%%%%%%%%%%%%%%%%%%%%%%%%%%%%%%%%%%%%%%%%%%%%%%%%%%%%%%%%%%%%%%%%%%


function P2
	r1 = 400+6378;
	r2 = 382900;
	mu = 3.985e5;
	
%% Quasi-Hohmann
	[a_H,t_H,dv_1, dv_2, dv_totH] = HohmannTransfer(r1,r2,mu);

%% Quasi-Bi-Elliptic
	ri = 3*0.3844e6;
	[a_1,a_2,dv_1,dv_i,dv_2,dv_totB,R,t1,t2,ttot] = BiEllipticTransfer(r1,r2,ri,mu);
	(1-(dv_totB/dv_totH))*100
end

%%%%%%%%%%%%%%%%%%%%%%%
% Bi-Elliptic Transfer
%%%%%%%%%%%%%%%%%%%%%%%

function [a_1,a_2,dv_1,dv_i,dv_2,dv_tot,R,t1,t2,ttot] = BiEllipticTransfer(r1,r2,ri,mu)
	a_1 = (r1+ri)/2;
	a_2 = (ri+r2)/2;

	dv_1 = sqrt(mu*((2/r1)-(1/a_1))) - sqrt(mu/r1);
	dv_i = sqrt(mu*((2/ri)-(1/a_2))) - sqrt(mu*((2/ri)-(1/a_1)));
	dv_2 = sqrt(mu/r2) - sqrt(mu*((2/r2)-(1/a_2)));
	dv_tot = abs(dv_1) + abs(dv_2) + abs(dv_i);

	R = r2/r1;

	t1 = pi*sqrt((a_1^3)/mu);
	t2 = pi*sqrt((a_2^3)/mu);
	ttot = t1+t2;
end
\end{lstlisting}
\newpage
%%%%%%%%%%%%%%%%%%%%%%%%%%%%%%%%%%%%%%%%%%%
%%%%%%%%%%%%%%%%%%%%%%%%%%%%%%%%%%%%%%%%%%%
\begin{lstlisting}
%%%%%%%%%%%%%%%%%%%%%%%%%%%%%%%%%%%%%%%%%%%%%%%%%%%%%%%%%%%%%%%%%%%%%%%%%%%
%% Problem 3A
%%%%%%%%%%%%%%%%%%%%%%%%%%%%%%%%%%%%%%%%%%%%%%%%%%%%%%%%%%%%%%%%%%%%%%%%%%%

function P3A
	% Program to optimize two-stage rocket mass
	x0=[20000,5000,3000];
	%options=optimset('LargeScale','off','display','iter');
	options=optimset('LargeScale','off');
	x=fmincon(@objfun3,x0,[],[],[],[],[],[],@confuneq3,options);
	x(1)
	x(2)
	x(3)
	f=objfun3(x)
	[c,ceq]=confuneq3(x);
end

% objfun3.m
% Objective function (total mass) for optimal rocket problem
function f=objfun3(x)
	mp=1000;
	f=x(1)+x(2)+x(3)+mp;
end

% confuneq3.m
% constraint equation for optimal rocket problem
function [c,ceq]=confuneq3(x)
	c1 = 3500;
	c2 = 3800;
	c3 = 4100;
	e1 = 0.10;
	e2 = 0.12;
	e3 = 0.09;
	mp = 1000;
	vfinal = 8500;

	ceq =   c1*log((x(1)+x(2)+x(3)+mp)/(e1*x(1)+x(2)+x(3)+mp))+...
	        c2*log((x(2)+x(3)+mp)/(e2*x(2)+x(3)+mp))+...
	        c3*log((x(3)+mp)/(e3*x(3)+mp))-vfinal;
	c=[];
end
\end{lstlisting}
\newpage
%%%%%%%%%%%%%%%%%%%%%%%%%%%%%%%%%%%%%%%%%%%
%%%%%%%%%%%%%%%%%%%%%%%%%%%%%%%%%%%%%%%%%%%
\begin{lstlisting}
%%%%%%%%%%%%%%%%%%%%%%%%%%%%%%%%%%%%%%%%%%%%%%%%%%%%%%%%%%%%%%%%%%%%%%%%%%%
%% Problem 3B
%%%%%%%%%%%%%%%%%%%%%%%%%%%%%%%%%%%%%%%%%%%%%%%%%%%%%%%%%%%%%%%%%%%%%%%%%%%

function P3B
% Program to optimize two-stage rocket mass
	x0=[20000,5000,3000];
	%options=optimset('LargeScale','off','display','iter');
	options=optimset('LargeScale','off');
	x=fmincon(@oobjfun3,x0,[],[],[],[],[],[],@cconfuneq3,options);
	x(1)
	x(2)
	x(3)
	f=oobjfun3(x)
	[c,ceq]=cconfuneq3(x);
end

% objfun3.m
% Objective function (total mass) for optimal rocket problem
function f=oobjfun3(x)
	mp=1000;
	f=x(1)+x(2)+x(3)+mp;
end

% confuneq3.m
% constraint equation for optimal rocket problem
function [c,ceq]=cconfuneq3(x)
	c1 = 3500;
	c2 = 3800;
	c3 = 4100;
	e1 = 0.10;
	e2 = 0.12;
	e3 = 0.09;
	mp = 1000;
	vfinal = 8500;
	g = 9.81;
	t = 90;

	ceq =   c1*log((x(1)+x(2)+x(3)+mp)/(e1*x(1)+x(2)+x(3)+mp))-g*t+...
	        c2*log((x(2)+x(3)+mp)/(e2*x(2)+x(3)+mp))+...
	        c3*log((x(3)+mp)/(e3*x(3)+mp))-vfinal;
		
	c=[];
end
\end{lstlisting}

\end{document}