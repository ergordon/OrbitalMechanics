\documentclass[11pt]{article}
\usepackage[utf8]{inputenc} % Para caracteres en espa�ol
\usepackage{amsmath,amsthm,amsfonts,amssymb,amscd}
\usepackage{multirow,booktabs}
\usepackage[table]{xcolor}
\usepackage{fullpage}
\usepackage{lastpage}
\usepackage{enumitem}
\usepackage{multicol}
\usepackage{fancyhdr}
\usepackage{mathrsfs}
\usepackage{pdfpages}
\usepackage{wrapfig}
\usepackage{setspace}
\usepackage{esvect}
\usepackage{calc}
\usepackage{multicol}
\usepackage{cancel}
\usepackage{graphicx}
\graphicspath{ {pictures/} }
\usepackage[retainorgcmds]{IEEEtrantools}
\usepackage[margin=3cm]{geometry}
\usepackage{amsmath}
\newlength{\tabcont}
\setlength{\parindent}{0.0in}
\setlength{\parskip}{0.05in}
\usepackage{empheq}
\usepackage{framed}
\usepackage[most]{tcolorbox}
\usepackage{xcolor}
\colorlet{shadecolor}{orange!15}
\parindent 0in
\parskip 12pt
\geometry{margin=1in, headsep=0.25in}
\theoremstyle{definition}
\newtheorem{defn}{Definition}
\newtheorem{reg}{Rule}
\newtheorem{exer}{Exercise}
% Two more packages that make it easy to show MATLAB code
\usepackage[T1]{fontenc}
\usepackage[framed,numbered]{matlab-prettifier}
\lstset{
	style = Matlab-editor,
	basicstyle=\mlttfamily\small,
}
\newtheorem{note}{Note}
\begin{document}  
\setcounter{section}{0}
\thispagestyle{empty}

\begin{lstlisting}
function p1
r1 = 1;
r2 = 5.2;

%% Part B
mu = 1.327e11; % Standard Gravitational Parameter of the Sun [km^3/s^2]
th = 150;

c = sqrt(r1^2 +r2^2 - 2*r1*r2*cosd(th));
s = (r1 + r2 + c)/2;
a_m = s/2;

% Calculate t_m
alpha_m = 2*asin(sqrt(s/(2*a_m)));
beta_m = 2*asin((sqrt((s-c)/(2*a_m))));
t_m = ((sqrt((au2km(s))^3/8))*(pi-beta_m + sin(beta_m)))/sqrt(mu);

% Calculate t_f
a=5; % Semi-major Axis [au]
alpha_o = 2*asin(sqrt(s/(2*a)));
beta_o = 2*asin((sqrt((s-c)/(2*a))));
t_f = (sqrt((au2km(a)^3)/mu)*(alpha_o-beta_o - sin(alpha_o) + sin(beta_o)));

% Calculate t_f^#
alpha_1 = 2*pi - alpha_o;
beta_1 = beta_o;
t_fup = (sqrt((au2km(a)^3)/mu)*(alpha_1-beta_1 - sin(alpha_1) + sin(beta_1)));

% Calculate t_p 
t_p = (sqrt(2)/(3*sqrt(mu)))*(au2km(s)^(3/2) - sign(sind(th))*(au2km(s)-au2km(c))^(3/2));

%% Part C

gamma = asind((r2*sind(th))/c);
ui = [1;0];
uc = [-cosd(gamma); sind(gamma)];

A_o = sqrt(1/(4*a))*cot(alpha_o/2);
B_o = sqrt(1/(4*a))*cot(beta_o/2);
v1_o = (B_o+A_o)*uc + (B_o-A_o)*ui;

A_1 = sqrt(1/(4*a))*cot(alpha_1/2);
B_1 = sqrt(1/(4*a))*cot(beta_1/2);
v1_1 = (B_1+A_1)*uc + (B_1-A_1)*ui;


%% Part E
p = ((4*a*(s-r1)*(s-r2))/(c^2))*(sin((alpha_o+beta_o)/2))^2;
p_tild = ((4*a*(s-r1)*(s-r2))/(c^2))*(sin((alpha_1+beta_1)/2))^2;
e = sqrt(1-(p/a));
end
\end{lstlisting}

\newpage
\begin{lstlisting}
function p2
	r1 = 1; %Earth [au]
	a_mars = 1.5237; % Mars [au]
	e_mars = 0.0934; f = 210; th = 120; % deg
	r2 = (a_mars*(1-e_mars^2))/(1+e_mars*cosd(f));
%% Part A
	c = sqrt(r1^2 +r2^2 - 2*r1*r2*cosd(th));
	s = (r1 + r2 + c)/2;
	a_m = s/2;
	mu = 1.327e11; % Standard Gravitational Parameter of the Sun [km^3/s^2]
	% Calculate t_m
	beta_m = 2*asin((sqrt((s-c)/(2*a_m))));
	t_m = ((sqrt((au2km(s)^3)/8))*(pi-beta_m + sin(beta_m)))/sqrt(mu);
%% Part B
	p = ((4*a_m*(s-r1)*(s-r2))/(c^2))*(sin((pi+beta_m)/2))^2;
	e = sqrt(1-(p/a_m));
%% Part C
	T_mars = 2*pi*sqrt((au2km(a_mars)^3)/mu); % Mars Orbital Period [s]
	syms E2 M tx f1
	f2 = 210;
	E2 = vpa(solve(tand(f2/2) == sqrt((1+e_mars)/(1-e_mars))*tan(E2/2),E2));
	M2 = solve(M == E2 - e_mars*(sin(E2)),M);
	%Time of mars at departure past perigee
	tx = solve(M2 == sqrt(mu/au2km(a_mars)^3)*(t_m+tx),tx);
	M1 = sqrt(mu/au2km(a_mars)^3)*tx;
	[iterations, values] = KeplersEqnNewtonMethod(M1,e_mars,M1,10^-14);
	E1 = values(end);
	f1 = rad2deg(double(vpa(solve(tan(f1/2) == sqrt((1+e_mars)/(1-e_mars))*tan(E1/2),f1))));
	b_mars = a_mars*sqrt(1-e^2);
	ra = [a_mars*cos(E1)-e_mars; b_mars*sin(E1);0];
	%OR
	%Define a new elapsed time past epoch.
	t_apogee = T_mars/2; %Time at apogee
	t_o = t_apogee-(T_mars + tx);
	a = au2km(a_mars);
	r_o = [a*(1+e_mars); 0; 0];
	v_o = [0; sqrt(mu*((2/norm(r_o))-(1/a))); 0];
	H_o = mu/norm(r_o)^3;
	P_o = 0;
	rb = km2au(r_o*(1-(t_o^2/2)*H_o) + v_o*(t_o - (t_o^3/6)*H_o));
%% Part D
	gamma = asind((r2*sind(th))/c);
	ui = [0;1];
	uc = [-cosd(gamma); -sind(gamma)]
	A = sqrt(1/(4*a_m))*cot(pi/2);
	B = sqrt(1/(4*a_m))*cot(beta_m/2);
	v1 = (B+A)*uc + (B-A)*ui;
%% Part E
	deltaV1 = v1 - [-1;0];
end
\end{lstlisting}
\newpage
\begin{lstlisting}

function p3
r1 = 9.538; % Saturn [au]
r2 = 1.523; % Mars [au]
th = 90; % deg

%% Part A
c = sqrt(r1^2 +r2^2 - 2*r1*r2*cosd(th));
s = (r1 + r2 + c)/2;
a_m = s/2;
 mu = 1; % Standard Gravitational Parameter of the Sun [AU^3/TU^2]

 %% Part B: Calculate t_m
alpha_m = 2*asin(sqrt(s/(2*a_m)));
beta_m = 2*asin((sqrt((s-c)/(2*a_m))));
t_m = ((sqrt((s)^3/8))*(pi-beta_m + sin(beta_m)))/sqrt(mu);

%% Part C: Calculate t_p 
t_p = (sqrt(2)/(3*sqrt(mu)))*(s^(3/2) - sign(sind(th))*(s-c)^(3/2));

%% Part D
tf = 67.12; % TU
a=fzero(@lambert,10);

%% Part F
alpha = 2*pi - real(2*asin(sqrt(s/(2*a))));
beta = 2*asin((sqrt((s-c)/(2*a))));
p = ((4*a*(s-r1)*(s-r2))/(c^2))*(sin((alpha+beta)/2))^2;
e = sqrt(1-(p/a));

%% Part G
gamma = asind((r2*sind(th))/c);
ui = [1;0];
uc = [-cosd(gamma); sind(gamma)];
A = sqrt(1/(4*a))*cot(alpha/2);
B = sqrt(1/(4*a))*cot(beta/2);
v1 = (B+A)*uc + (B-A)*ui;
aH = (r1+r2)/2;
deltaV1 = sqrt(mu*((2/r1)-(1/aH))) - sqrt(mu/r1);
end

function f = lambert(a)
r1 = 9.538; % Saturn [au]
r2 = 1.523; % Mars [au]
th = 90; % deg
c = 9.658828759223345;
s = 10.359914379611673;
alpha = 2*pi + real(2*asin(sqrt(s/(2*a))));
beta = 2*asin((sqrt((s-c)/(2*a))));
tf = 10.69;
f = tf - (a^(3/2)/(2*pi))*(alpha-beta-sin(alpha)+sin(beta));
end
\end{lstlisting}
\newpage
\begin{lstlisting}
function p4
r1 = 1; %Earth [au]
r2 = 1.524; % Mars [au]
a = 1.36; % [au]
a_m = 1.14; % [au]
th1 = 107; % deg
th2 = 253; % deg
mu = 1;
%% Part A

c1 = sqrt(r1^2 +r2^2 - 2*r1*r2*cosd(th1));
s1 = (r1 + r2 + c1)/2;

c2 = sqrt(r1^2 +r2^2 - 2*r1*r2*cosd(th2));
s2 = (r1 + r2 + c2)/2;

c = c1;
s = s1;
mu = 1.327e11; % Standard Gravitational Parameter of the Sun [km^3/s^2]
%mu = 1; % Standard Gravitational Parameter of the Sun [AU^3/TU^2]

% Calculate t_m
beta_m = 2*asin((sqrt((s-c)/(2*a_m))));
t_m = ((sqrt((au2km(s))^3/8))*(pi-beta_m + sin(beta_m)))/sqrt(mu);
fprintf('t_m   : %f days\n',t_m/(60*60*24));

% Calculate t_f
alpha_o = 2*asin(sqrt(s/(2*a)))
beta_o = 2*asin((sqrt((s-c)/(2*a))))
t_f = ((sqrt((au2km(s))^3/8))*(alpha_o-beta_o - sin(alpha_o) + sin(beta_o)))/sqrt(mu);
fprintf('t_f   : %f days\n',t_f/(60*60*24));

% Calculate t_f^#
alpha_1 = 2*pi - alpha_o;
beta_1 = beta_o;
t_fup = ((sqrt((au2km(s))^3/8))*(alpha_1-beta_1 - sin(alpha_1) + sin(beta_1)))/sqrt(mu);
fprintf('t_f^# : %f days\n',t_fup/(60*60*24));

end
\end{lstlisting}
\end{document}